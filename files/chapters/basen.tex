\section{Basen -- Laugen}
\cVersuch{2}{\acl{11} in \acl{H2O} mit Phenolphthalein}
\begin{description}
   \item[Aufbau und Durchführung:] Wir entfernten die äußere oxidierte Schicht bei einem Stück \ac{11}, anschließend legten wir dieses in ein breites mit \ac{H2O} und Phenolphthalein gefülltes Becherglas.
   \item[Ergebnis:] Das Stückchen \ac{11} zischte, flitzte herum und es entstand weißer Rauch. Es wurde kleiner, nahm die Form einer Kugel (wegen der geringen Oberfläche) an und war schließlich völlig aufgelöst.
   Das \ac{H2O} färbte sich rötlich $\rightarrow$ Lauge.\\
   \ce{2Na + H2O ->} $\mathrm{\stackrel{\RM{2}}{Na}\stackrel{\RM{2}}{OH}}$ \ce{+  H2}\\
   \ce{2Na + 2H2O -> 2Na^+ + 2OH^- + H2} (Dissoziationsgleichung)
\end{description}
\begin{align*}
\cee{2K + 2H2O -> 2KOH +H2\\
2K + 2H2O -> 2K^+ + 2OH^- + H2}
\end{align*}

\begin{tabular}{lcc}
Alkalibasen & \entspricht & Basen mit (unedlen) Metallen aus der 1. Hauptgruppe. \\
Erdalkali Basen & \entspricht & Basen mit (unedlen) Metallen aus der 2. Hauptgruppe. \\
\end{tabular}

\cVersuch{2}{\acl{20} in \acl{H2O}}
\begin{description}
   \item[Aufbau und Durchführung:] Wir gaben \ac{20} in \ac{H2O} mit Lackmus als Indikator.
   \item[Ergebnis:] \ac{H2} Gas tritt aus (blubbert). Dies war erkennbar, als es angezündet wurde (Knallgasreaktion).
   Die Lackmuslösung färbte sich dunkelgrau-blau $\rightarrow$ es ist eine Lauge.
\end{description}

\cVersuch{2}{Oxidiertes \acl{12} in \acl{H2O}}
\begin{description}
   \item[Aufbau und Durchführung:] Wir verbrannten \ac{12} und gaben die Verbrennungsrückstände \ac{MgO} in ein Reagenzglas mit Lackmuslösung.
   \item[Ergebnis:] Je mehr \ac{MgO} in die Lackmuslösung kam desto blauer wurde diese $\rightarrow$ Lauge\\
   \ce{MgO + 2H2O -> }$\mathrm{\stackrel{\RM{2}}{Mg}}$\ce{(OH)2}
\end{description}
\vspace{-0.4cm}
\begin{center}
\includegraphics[width=1cm]{files/pst-labo/MgOundH2O-pics}
\end{center}

Elektronegativitätswert: $1,2$ \ac{12}\\
EN: $3,5$ \ac{O2}\\
$\mathrm{\Delta}$EN: $2,3 > 1,7 \rightarrow $ Ionenbindung\\
$\mathrm{\Delta}$EN = 0 reine Atombindung (nicht dissoziiert)\\
$\mathrm{\Delta}$EN $< 1,7$ polare Atombindung (nicht dissoziiert)\\
$\mathrm{\Delta}$EN $> 1,7$ Ionenbindung

\ce{MgO^2+ + O^2- + H2O -> }$\mathrm{\stackrel{\RM{2}}{Mg}}$\ce{^{2+} + 2OH^+}

\textbf{Ausnahmen:} \ce{NH4OH <=> NH4^+ + OH^-}

\hspace{3.15cm} \acl{NH4+} Hydroxidion

Viele Basenlösungen/Laugen lassen sich herstellen, indem man die Metalle \ac{11}, \ac{19}, \ac{20} direkt mit Wasser reagieren lässt.
Als Nebenprodukt entsteht Wasserstoff.

Auch die Verbrennungsprodukte vieler Metalle (Metalloxide) reagieren mit Wasser zu Laugen (kein Nebenprodukt).

Unedle Metalle (\ac{11}, \ac{19}, \ac{20}) \ce{+ Wasser -> Lauge + H} \\
\ce{Matalloxid + Wasser -> Lauge}

\subsection{Einige ausgewählte Basen (Laugen)}
\begin{center}
\begin{tabular}{|c|c|c|}
\hline \textsc{Stoff} & \textsc{Dissoziationsgleichungen} & \textsc{wässrigen Lösung} \\
\hline Natriumhydroxid \acs{NaOH} & \ce{NaOH <=> Na^+ + OH^-} & \acl{NaOH} \\
\hline \acf{KOH} & \ce{KOH <=> K^+ + OH^-} & Kalilauge \\
\hline \acf{Ca(OH)2} & \ce{Ca(OH)2 <=> Ca^2+ + 2OH^-} & Kalziumlauge\\
& & Kalkwasser \\
& & Löschkalk \\
\hline
\end{tabular}
\end{center}

\subsection{Definition}
Basen sind Stoffe, die in wässriger Lösung oder in der Schmelze in
positiv geladene Metallionen (Kationen)
und negativ geladene Hydroxidionen (Anionen) zerfallen.
