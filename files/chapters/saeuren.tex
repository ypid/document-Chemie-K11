\section{Säuren}
\subsection{Die wichtigsten Säuren}
\begin{tabular}{|c|c|c|c|c|}
\hline & \textsc{\acl{H2SO4}} & \textsc{\acl{HNO3}} & \textsc{\acl{HCl}} & \textsc{\acl{H2CO3}} \\
\hline \textsc{Formel} & \acs{H2SO4} & \acs{HNO3} & \acs{HCl} & \acs{H2CO3} \\
\hline \textsc{Eigen-} & farblos & farblos & farblos & farblos \\
\hline \textsc{schaften} & flüssig & flüssig & flüssig & flüssig \\
\hline & sehr stark Ätzend & stark Ätzend & Ätzend & \textdiscount \\
\hline & gesundheits- & gesundheits- & gesundheits- & \textdiscount \\
& schädlich & schädlich & schädlich & \\
\hline \textsc{Sulfatname} & Sulfat & Nitrate & Chloride & Karbonate \\
\hline \textsc{Beispiel} & \acl{CuSO4} & \acl{Na+} & \acl{NaCl} & \acl{CaCO3} \\
\hline \textsc{Als Formel} & \acs{CuSO4} & \acs{Na+} & \acs{NaCl} & \acs{CaCO3} \\
\hline \textsc{Restion} & \ce{SO4^2-} & \ce{NO3^-} & \ce{Cl^-} & \ce{CO3^2-} \\
\hline
\end{tabular}

\cVersuch{2}{Zucker in \acl{H2SO4}}
\begin{description}
   \item[Aufbau und Durchführung:] Es befand sich \ac{H2SO4} in einem Becherglas und wir gaben Zucker dazu.
   \item[Ergebnis:] Der Zucker wurde gelb, nach kurzer Zeit schwarz. Dies zeigt die starke hygroskopische Wirkung der Schwefelsäure. \\
   Die stärkerer Säure verdrängt die schwächere Säure aus ihren Salzen.
\end{description}

\cVersuch{2}{Schwefelverbrennung}
\begin{description}
   \item[Aufbau und Durchführung:] Wir füllten einen Erlenmeyerkolben mit \ac{H2O}, gemischt mit Lackmus und gaben \ac{16} auf einen Verbrennungslöffel, anschließend zündeten wir das \ac{16} mit einem Bunsenbrenner an und steckten den Verbrennungslöffel (luftdicht abgeschlossen) in den Erlenmeyerkolben.
   \item[Ergebnis:] Als das \ac{16} angezündet wurde, brannte es mit kleiner blauer Flamme und nach dem es eine gewisse Zeit im Erlenmeyerkolben war, färbte sich die Lackmuslösung hellrot.
   Der Rauch war also säurehaltig.
\end{description}
\ce{Nichtmetalloxid + H2O ->T[reagieren][zu]}Säurelösungen\\
Ausnahme: \ce{HCl \aggre{g} + H2O ->T[reagieren][zu] HCl \aggre{l}} \\

Beispiel:
\vspace{-1.14cm}
\begin{align*} \cee{SO3 + H2O &->T[reagieren][zu] H2SO4 \\
   CO2 + H2O &->T[reagieren][zu] H2CO3\\
   4NO2 + 2H2O + O2 &->T[reagieren][zu] 4HNO3}
\end{align*}

% \cVersuch{3}{Stromleitende Kohlensäure}


\ce{H2CO3 <=> 2H+ + CO3^2-}
\vspace{0.4cm}

\begin{tikzpicture}[color = {black},scale = 1]
   \draw (0.3,0) -- (0,0) -- (0,2) -- (0.3,2); %   \draw (1.9,1.8) -- (1.7,2.2);
   \draw (2.7,0) -- (3.0,0) -- (3.0,2) -- (2.7,2);
   \draw (1.3,1.7) -- (1.7,1.7);
   \node[scale = 1] at (3.3,1.8) {$+$};
   \node[scale = 1] at (1.5,1.4) {O};
   \node[scale = 1] at (2.3,0.5) {H};
   \node[scale = 1] at (1.5,0.5) {H};
   \node[scale = 1] at (0.7,0.5) {H};
   \draw (1.5,1.1) -- (1.5,0.8);
   \draw (1.3,1.2) -- (0.7,0.8);
   \draw (1.7,1.2) -- (2.3,0.8);
   \node[scale = 1] at (5,1.4) {\ce{H3O^+}};
\end{tikzpicture}

Johannes Nicolaus Brønsted: \ce{H2CO3 + 2H2O -> 2H2O+ + CO3^2-}
\vspace{0.3cm}
\begin{description}
   \item[\acl{HCl}:] Arrhenius: \ce{HCl <=> H+ + Cl-} \\
   Brønsted: \ce{HCl + H2O <=> H3O+ + Cl-} (Oxeoniumion)
   \item[\acl{HNO3}:] Arrhenius: \ce{HNO3 <=> H+ + NO3^-} \\
   Brønsted: \ce{HNO3 + H2O <=> H3O+ + NO3^-}
   \item[\acl{H2SO4}:] Arrhenius: \ce{H2SO4 <=> 2H+ + SO4^2-} \\
   Brønsted: \ce{H2SO4 + 2H2O <=> 2H3O+ +SO4^2-}
   \item[\acl{H3PO4}:] Arrhenius: \ce{H3PO4 <=> 3H+ + PO4^3-} \\
   Brønsted: \ce{H3PO4 + 3H2O <=> 3H3O+ + PO4^3-}
   \item[schweflige Säure:] Arrhenius: \ce{H2SO3 <=> 2H^+ + SO3^2-} \\
   Brønsted: \ce{H2SO3 + 2H2O <=> 2H3O^+ + SO3^2-}
   \item[\aclu{H2S}:] Arrhenius: \ce{H2S <=> 2H^+ + S^2-} \\
   Brønsted: \ce{H2S + 2H2O <=> 2H3O^+ + S^2-}
\end{description}

\subsection{Dissoziation}
Zerfall von Stoffen in wässriger Lösung oder in der Schmelze, in frei bewegliche Ionen.


\subsection{Nachweis von Säurerestionen mittels Fällungsreaktion}
Die Fällungsreaktion ist eine Reaktion, bei der Ionen eines schwer löslichen Salzes in einer Lösung
zusammentreten, sodass diese Salze als Niederschlag ausfallen.

Lösung mit positiv und negativ geladenen Molekülen.

$\rightarrow$ Produkt: ein schwer- und ein leichter löslicher Stoff.

\vspace{0.3cm}
\begin{center}
\begin{tabular}{|c|c|c|c|}
\hline \textsc{Säure-} & \textsc{aussäuern} & \textsc{Nachweis-} & \textsc{Ionengleichung} \\
\textsc{restion} & \textsc{mit} & \textsc{mittel} & \textsc{Beobachtung} \\
\hline \ce{SO4^2-} & \textit{v} \ce{HCl} & \ce{BaCl2 \aggre{l}} & \ce{SO4^2- + Ba^2+ -> BaSO4 v} weißer Niederschlag \\
\hline \ce{Cl^-} & \textit{v} \ce{HNO3} & \ce{AgNO3 \aggre{l}} & \ce{Cl^- + Ag^+ -> AgCl v} weißer NS \\
\hline \ce{CO3^2-} & \textdiscount & \ce{Ca(OH)2 \aggre{l}} & \ce{CO3^2- + Ca^2+ -> CaCO3 v}weißer NS \\
\hline \ce{CO3^2-} & \textdiscount & \ce{Ba(OH)2 \aggre{l}} & \ce{CO3^2- + Ba^2+ -> BaCO3 v}weißer NS \\
\hline \ce{NO3^-} & \textit{v} \ce{H2SO4} & \ce{FeSO4 \aggre{l}} & Ringprobe, violetter Ring \ce{Fe(NO)SO4}\\
\hline
\end{tabular}
\end{center}

\subsection{Definition nach Arrhenius}
Säuren zerfallen in wässrige Lösungen in positiv geladenen Wasserstoffionen und negativ geladene
Säurerestionen.
