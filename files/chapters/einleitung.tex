\poemtitle{Aus Goethes Faust -- Homunculus}
\poemlines{4}
\settowidth{\versewidth}{Durch Mischung – denn auf Mischung kommt es an -}
\begin{verse}[\versewidth]
    Es leuchtet! seht! -- Nun läßt sich wirklich hoffen,\\
    Daß, wenn wir aus viel hundert Stoffen\\
    Durch Mischung -- denn auf Mischung kommt es an -\\
    Den Menschenstoff gemächlich komponieren,\\
    In einen Kolben verlutieren\\
    Und ihn gehörig kohobieren,\\
    So ist das Werk im stillen abgetan.\\
    Es wird! die Masse regt sich klarer!\\
    Die Überzeugung wahrer, wahrer:\\
    Was man an der Natur Geheimnisvolles pries,\\
    Das wagen wir verständig zu probieren,\\
    Und was sie sonst organisieren ließ,\\
    Das lassen wir kristallisieren.\\
\end{verse}
\attrib{Johann Wolfgang von Goethe}
\newpage

\section{Einleitung}
Thema dieser Epoche sind Salze, Säuren und Basen (Laugen).

\subsection{Für was werden Salze, Säuren und Basen benötigt?}
\begin{tabularx}{\textwidth}{|X|X|X|}
\hline \textsc{Salze} & \textsc{Säuren} & \textsc{Basen (Laugen)} \\
\hline Lebensmittel (Wasserhaushalt, fürs Gehirn, im Körper) & Medizin & Seifenherstellung \\
\hline \acl{NaCl}: Medizin & \acl{HCl} im Magen (0,15 \% - 0,2 \%)  & Medizin \\
\hline physiologische \acl{NaCl}-Lösung (9g \acl{NaCl} pro Liter \acl{H2O}) als Blutersatz & \acl{82}-Akku & Reinigung \\
\hline Bauwesen & Lösungsmittel & Ernährung \\
\hline Kalkstein, Marmor, Gips & Sprengstoffherstellung & Neutralisation \\
\hline Vorkommen: Gebirge, Meere & Ernährung & \\
\hline Parma-Industrie & Parma-Industrie & Düngemittel \\
\hline
\end{tabularx}
